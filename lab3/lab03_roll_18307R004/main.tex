\documentclass[11pt]{article}
\usepackage{multicol}
\usepackage{graphicx}
\usepackage{todonotes}
\usepackage{tcolorbox}

\usepackage{anysize,color,float,graphicx,url,comment,hyperref,ulem}
\usepackage{hyperref}
\hypersetup{
    colorlinks=true,
    linkcolor=blue,
    filecolor=magenta,      
    urlcolor=red,
}
\usepackage{amsmath, amsfonts, amssymb}
\usepackage{enumitem}
\usepackage{HW}
\includecomment{comment}
\excludecomment{studentSpace}
\includecomment{studentSpace}

\includecomment{answer}
\excludecomment{answer}
\includecomment{comment}
\newcommand{\blank}{
\underline{\hspace{1.5cm}\hspace{1.5cm}}
}
\newcommand{\norm}[1]{\left\lVert#1\right\rVert}

\begin{document}
\titleline{\semester}
\exhead{Problem Set: Due: 10:00 PM,  18-Feb}
\begin{itemize}
\item Please write (only if true) the honor code. If you used any
  source (person or thing) explicitly state it. 

\item Important: This is an INDIVIDUAL assignment.

\item Always provide a brief explanation. (The length of the
  explanation required has been forecasted with the amount of space
  provided.)
  
\item Submit following files in folder name lab03\_roll\_XX :
	\begin{enumerate}
      \item readme.txt (case sensitive name). This \emph{text} file
        contains identifying information, honor code, links to
        references used 
      \item ReflectionEssay.pdf is optional but a brief one would be nice.
      \item lab03\_roll\_XX.pdf (includes all solutions). 
      \item All relevant tex source (and images only if necessary). No
        other junk files, please.
     \end{enumerate}
\end{itemize} 

\hrulefill
\begin{enumerate}

\begin{tcolorbox}
\item  State whether or not the following points are the same and explain why.
\begin{multicols}{2}
\begin{enumerate}[]
\item
$A[2, -1, 3]$, $B[4,-2,6]$
\item
$A[\sqrt{2}/2, -1,0]$, $B[1,-\sqrt{2},0]$
\end{enumerate}
\end{multicols}
\begin{studentSpace}
%\vspace{3cm}
\begin{itemize}
    \item In terms of the coordinate geometry the given pair of points are distinct in both the cases.
    \item While if we consider the points in a projective plane then the given pair of points are same as they lie on a single ray originating from the origin. This is true for both the cases a and b.
\end{itemize}
\end{studentSpace}

\end{tcolorbox} \begin{tcolorbox}
\item In projective three-space, what are the standard homogeneous
  coordinates of (a) the origin and (b) ideal points determined by the
  intersections of the extensions of the coordinate axes and the ideal
  plane?


\begin{studentSpace}
%\vspace{3cm}
In projective three-space the standard homogeneous coordinates of,
\begin{enumerate}[a]
    \item  the origin is represented by (0, 0, 1).
    \item the horizontal and vertical directions are represented by points (1,0,0) and (0,1,0) respectively.
\end{enumerate}
\end{studentSpace}
\end{tcolorbox}

\begin{tcolorbox}
\item Write standard homogeneous coordinates for the points specified
  in uppercase characters. (Use left and right to distinguish.)

\begin{minipage}{0.45\textwidth}
\includegraphics[width=\textwidth, height=.5\textheight,keepaspectratio]{image2.png}
\end{minipage} \hfill
\begin{minipage}{0.45\textwidth}
\includegraphics[width=\textwidth, height=.5\textheight,keepaspectratio]{image1.png}
\end{minipage}

\begin{studentSpace}
%\vspace{3cm}
\begin{minipage}{0.45\textwidth}
\begin{itemize}
    \item left
    \begin{itemize}
        \item $O[0,1]$
        \item $A[-0.5,1]$
        \item $B[4,1]$
        \item $C[6,1]$
        \item $D[6.5,1]$
        \item $E[1,0]$
    \end{itemize}
\end{itemize}
\end{minipage} \hfill
\begin{minipage}{0.45\textwidth}
\begin{itemize}
    \item right
    \begin{itemize}
        \item $A[0,0,1]$
        \item $B[0,2,1]$
        \item $C[3,1,1]$
        \item $D[1,1,0]$
        \item $E[-1,4.5,1]$
        \item $F[-1,1,0]$
        \item $G[-3,4,1]$
        \item $H[-4,3,1]$
        \item $I[-1,1,1]$
        \item $J[-4,-2,1]$
        \item $K[1,-4,1]$
        \item $L[1.5,-0.5,1]$
         \item $M[0,1,0]$
        
    \end{itemize}
\end{itemize}
\end{minipage}
\end{studentSpace}



\end{tcolorbox} \begin{tcolorbox}
\item Which of the following points lie on the line $3p_1 -2p_2+5p_3 =
  0$?  Why?
\begin{multicols}{2}
\begin{enumerate}[]
\item
$A[1,1,2]$
\item
$B[4,1,-2]$
\end{enumerate}
\end{multicols}
\begin{studentSpace}
The point $p = [x,y,z,1]$ lies on the line $l=[a,b,c,d]$ iff $p^Tl = l^Tp = 0$. Therefore, $B[4,1,-2]$ lies on the given line.
\end{studentSpace}
\end{tcolorbox} \begin{tcolorbox}
\item Write the coordinates of the lines that are the extensions to the
  projective plane of the following Euclidean lines.
\begin{multicols}{2}
\begin{enumerate}[]
\item
$3x + 2y = 6$
\item
$4x + 5y + 7 = 0$
\end{enumerate}
\end{multicols}
\begin{studentSpace}
The coordinates of the lines that are the extensions to the
  projective plane of the given Euclidean lines are,
  \begin{enumerate}[]
      \item $[3,2,-6]$
      \item $[4,5,7]$
  \end{enumerate}
\end{studentSpace}

\end{tcolorbox} \begin{tcolorbox}
\item Sketch each line in the projective plane whose equation is given.
\begin{multicols}{2}
\begin{enumerate}[]
\item
$2p_1 + 3p_2 + 5p_3 = 0$
\item
$3p_1 - 2p_2 - p_3 = 0$
\end{enumerate}
\end{multicols}
\begin{studentSpace}
\vspace{6cm}
\end{studentSpace}


\end{tcolorbox} \begin{tcolorbox}
\item  In each of the following cases, sketch the line determined by the two given points; then find the equation of the line.
\begin{multicols}{2}
\begin{enumerate}[]
\item
$A[3, 1, 2]$, $B[1,2,-1]$
\item
$A[2,1,3]$, $B[1,2,0]$
\end{enumerate}
\end{multicols}
\begin{studentSpace}
\vspace{6cm}
\end{studentSpace}


\end{tcolorbox} \begin{tcolorbox}
\item  Find the standard homogeneous coordinates of the point of
  intersection for each pair of lines. 
\begin{multicols}{2}
\begin{enumerate}[]
\item
$p_1 + p_2 - 2p_3 = 0, 3p_1 + p_2 + 4p_3 = 0$
\item
$p_1 + p_2 = 0, 4p_1 - 2p_2 + p_3 = 0$
\end{enumerate}
\end{multicols}
\begin{studentSpace}
The point of intersection $p$ of two lines $u_1$ and $u_2$ can be calculated by,
$$p = u_1 \times u_2$$
Hence the homogenous coordinates of the point of intersection for the given pair of lines are,
\begin{enumerate}[a]
    \item $[-3, 5, 1]$
    \item $[\frac{-1}{6}, \frac{1}{6}, 1]$
\end{enumerate}
\end{studentSpace}


\end{tcolorbox} \begin{tcolorbox}
\item  Determine which of the following sets of three points are collinear. 
\begin{multicols}{2}
\begin{enumerate}[]
\item
$A[1,2,1]$, $B[0,1,3]$, $[2,1,1]$
\item
$A[1,2,3]$, $B[2,4,3]$, $[1,2,-2]$
\end{enumerate}
\end{multicols}
\begin{studentSpace}
To determine whether three points $A, B, C$ are collinear, it would suffice to check if the determinant of the $3 \times 3$ matrix containing the points is zero.
\begin{minipage}{0.45\textwidth}
$\begin{vmatrix}
1 & 2 & 1\\
0 & 1 & 3\\
2 & 1 & 1
\end{vmatrix} = 8 \neq 0$
\end{minipage} \hfill
\begin{minipage}{0.45\textwidth}
$\begin{vmatrix}
1 & 2 & 3\\
2 & 4 & 3\\
1 & 2 & -2
\end{vmatrix} = 0$
\end{minipage}\\
Hence, the set of points $A[1,2,3]$, $B[2,4,3]$, $[1,2,-2]$ is collinear.
\end{studentSpace}

\end{tcolorbox} \begin{tcolorbox}
\item  Determine which of the following sets of three lines meet in a
  point. 
\begin{multicols}{2}
\begin{enumerate}[]
\item
$l[1,0,1], m[1,1,0], n[0,1,-1]$
\item
$l[1,0,-1], m[1,-2,1], n[3,-2,-1]$
\end{enumerate}
\end{multicols}
\begin{studentSpace}
To determine whether a set of three lines $l, m, n$ meet at a point, it would suffice to check if the determinant of the matrix \begin{bmatrix}
l & m & n
\end{bmatrix}is zero.\\
\begin{minipage}{0.45\textwidth}
$\begin{vmatrix}
1 & 0 & 1\\
1 & 1 & 0\\
0 & 1 & -1
\end{vmatrix} = 0$
\end{minipage} \hfill
\begin{minipage}{0.45\textwidth}
$\begin{vmatrix}
1 & 0 & -1\\
1 & -2 & 1\\
3 & -2 & -1
\end{vmatrix} = 0$
\end{minipage}\\
Accordingly, both the given set of three lines meet in a point.
\end{studentSpace}
\end{tcolorbox}

\end{enumerate}
\end{document}
